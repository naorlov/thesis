\documentclass[a4paper,12pt]{article}
\usepackage{header}
\usepackage{hyperref}

 % Цвета для гиперссылок
\usepackage{xcolor}
\usepackage{floatflt}
\usepackage{ulem}

% hack for indentation
\usepackage{indentfirst}
\usepackage{longtable}
\usepackage{graphicx}

\usepackage{chngcntr}
\counterwithin{figure}{section}
\counterwithin{table}{section}

\usepackage{caption}
\captionsetup[table]{position=t,justification=raggedleft,slc=off}

% hack for список литературы
\usepackage{etoolbox}
\patchcmd{\thebibliography}{\section*}{\section}{}{}
\fancyhf{}
\fancyhead[R]{\head}
\fancyhead[L]{\thepage}

\def\baselinestretch{1}

\makeatletter
\let\newtitle\@title
\let\newdate\@date
\makeatother

\linespread{1.3}

\begin{document}
    \input{sections/titlepage.tex}
    
    \pagestyle{plain}
    \tableofcontents
    
    \newpage

    \section{Введение}\label{sec:introduction}

    \subsection{Актуальность}\label{sec:relevance}

    \subsection{Цели и задачи}\label{sec:goals}

    \newpage

    \section{О MapReduce}\label{sec:whatismapred}

    \newpage
    
    \section{Технологии симуляции}

    \subsection{ROSS}

    \subsection{CODES}

    \subsection{SimGrid}

    \newpage

    \section{Обзор существующих решений}
    
    \subsection{MRSim}

    \subsection{MRperf}

    \subsection{MRSG}

    \subsection{YarnSim}

    \subsection{Mumak}

    \subsection{BTeHadoop2}

    \newpage

    \section{Мой вклад}

    \subsection{Архитектура}

    \subsection{Особенности}

    \subsection{Работа с симулятором}

    \newpage

    \section{Тестирование}

    \newpage

    \section{Заключение}

    \newpage

    \printbibliography

\end{document}
