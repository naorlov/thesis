\documentclass[../diploma.tex]{subfile}


\begin{document}
    
    \subsection{Обзор существующих решений}\label{sec:review:subsec:existing}

    Среди симуляторов распределенных систем можно найти представителей разных
    подходов к моделированию. 

    Среди дискретно-событийных симуляторов можно выделить нескольких
    представителей: Mumak \cite{mumak}, SimMR \cite{simmr}, MRperf
    \cite{mrperf}, MRSG \cite{mrsg}, YarnSim \cite{yarnsim} and MRSim
    \cite{mrsim}. Все они разделяют между собой одну идею -- они моделируют ядро
    MapReduce системы (планировщик, управление и квотирование ресурсами,
    распределенная файловая система), при этом характеристики времени выполнения
    MapReduce-задач предоставляются пользователем в соответствии с инструкцией
    разработчиков симуляторов. 

    В отличие от них, Song и другие в своей статье \cite{song} используют методы
    аналитического моделирования. Симулятор собирает метаинформацию о
    пользовательских задачах и использует ее в качестве входных данных для
    определенной авторами специальной модели времени выполнения задачи. 
    
    YARNSim предполагает, что все Map и Reduce шаги имеют одну функцию
    зависимости входных данных от времени, тем самым упрощая для себя разработку
    и оставляя пользователю симулятора только один параметр конфигурации --
    размер входных данных. 

    BTeHadoop2 \cite{baseline_model} является наиболее точным и полным
    симулятором среди всех представленных, показывая лучшую точностью и скорость
    работы симуляции. Авторы используют комбинированный подход
    дискретно-событийных симуляторов для моделирования внутренних процессов
    MapReduce системы, и аналитическую модель для определения времени выполнения
    отдельных шагов задачи. Для начала авторы запускают пользовательские
    программы на небольшом объеме данных, после чего вычисляют многочлен
    Лагранжа для получения модели времени выполнения. Затем, он использует эту
    информацию для определения времени, которое займет обработка данных
    соответствующими шагами MapReduce задачи.


    \subsubsection{Выводы}

    Существует много различных методов, подходов и технологий симуляции
    компьютерных систем. Выбор нужного подхода зависит от требований,
    предъявляемых к результату. Исходя из схожести задачи данной работы с
    существующими симуляторами MapReduce систем, при проектировании решения
    будут учитываться достоинства и недостатки различных симуляторов.
    
\end{document}


