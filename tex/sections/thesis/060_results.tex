\documentclass[../diploma.tex]{subfile}


\begin{document}
    \section{Мой вклад}\label{sec:results}

    В рамках выполнения дипломной работы было создано два программных модуля:
    \textit{mrss-analyzer} и \textit{mrss-simulator}.


    \subsection{Пререквизиты}

    Для работы с моим симулятором пользователю необходимо удостовериться, что
    выполненые следующие требования и ограничения:

    \begin{enumerate}
        \item Каждый шаг MapReduce задачи является бинарным исполняемым файлом,
              производящим чтение входных данных со стандартного ввода, и
              записывающим результат на стандартный вывод
        \item Шаги могут быть объединены в произвольную последовательность
        \item Для начального шага задан набор файлов, являющиийся эталонным и
              представляющий исходное распределение данных
    \end{enumerate}

    \subsection{Конфигурация задачи}

    В рамках дипломной работы был принят упрощенный формат настройки MapReduce
    задачи \textit{specfile}. Пользователь указывает следующие параметры:
    
    \begin{itemize}
        \item Версию конфигурационного файла
        \item Размер входных данных, подаваемых первому по порядку шагу вычислений
        \item Путь к исполняемым файлам для каждого шага вычислений
        \item Путь к входным данным для каждого шага вычислений
    \end{itemize}

    Пример описания приведет в Листинге \ref{lst:specfile} (в Приложении). 
    

    \subsection{Анализатор MapReduce-задачи}

    На основе конфигурационного файла проводится тестирование шагов указанной 
    MapReduce\hyp{}задачи. Результатом тестирования является модель времени
    исполнения и объем выходных данных шага задачи.

    Каждый шаг запускается несколько раз на каждом входном файле, которые
    указаны в конфигурационном файле (по умолчанию 10), после чего считается
    среднее значение среди 95-перцентиля всех значений. Это позволяет избавиться
    от выбросов в данных, которые неизбежно появляются из-за особенностей работы
    операционной системы. В итоге получается набор точек $(n_i, t_i)$, где $n_i$
    -- размер входных данных в строках, $t_i$ -- время исполнения шага,
    отвечающее соответствующему размеру входных данных.
    
    Затем происходит поиск подходящей модели. Для уменьшения пространства поиска
    модели рассматриваются только модели из следующего семейства функций:
    
    \begin{equation} 
        f(n) = \sum_{i = 0}^d a_in^i + \sum_{i=0}^d b_in^i\log(n)
    \end{equation} 

    Здесь параметризация семейства идет по параметру $d$ -- степени линейной
    части. $n$ является количеством входных строк. Для всех значений $d$ вплоть
    до верхнего предела, указанного пользователем (по умолчанию 10) происходит
    регрессия времени исполнения на коэффициенты $a_i, b_i$. 
    
    Данное семейство взято исходя из предположения, что для своей работы шаг задачи должен 

    Результатом работы этого этапа симуляции является текстовый файл с описанием
    функций времени исполнения и выходных данных для каждого шага
    MapReduce-задачи.

    \subsection{Симулятор MapReduce}
    
    Симулятор построен на базе фреймворка SimGrid \cite{simgrid}, позволяющего с
    легкостью описать распределенную систему в акторной модели. Симулятор
    работает с конфигурационным файлом, полученным от пользователя, файлом
    описания платформы и файлом результатов тестирования, полученным на этапе
    тестирования. 

    Для каждого шага симулятор создает <<задание>> на исполнение, состоящее из
    объема входных данных, функции зависимости времени выполнения и объема
    выходных данных для этого шага. Затем, для каждого сервера в эмулируемом
    кластере создается свой актор, ответственный за работу этого узла кластера.

    Каждый актор использует вызовы фреймворка SimGrid для симуляции исполнения
    шага MapReduce-задачи на узле, а именно чтение данных из файловой системы,
    выполнение программы и запись данных в файловую систему.

    В симуляторе поддержана функциональность использования собственноручно
    написанного алгоритма планирования вычисления. Для этого на случайном хосте
    симулируемого кластера создается \textit{координатор} -- специальный актор,
    чьей единственной задачей явлеяется отправка заданий другим акторам.




\end{document}


