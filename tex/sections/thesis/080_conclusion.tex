\documentclass[../../thesis.tex]{subfile}


\begin{document}
    \section{Заключение}
    \label{sec:final}


    Сложно переоценить важность симулятора MapReduce систем. Существование таких
    инструментов дает возможность создания новых алгоритмов планирования, а
    также постройки эффективных кластеров исходя из предполагаемой нагрузки. 

    В рамках данной работы был предложен новый метод симуляции MapReduce
    вычислений, комбинирующий дискретно-событийную модель и аналитическое
    моделирование. Цели и задачи выполнены практически полностью -- не удалось
    получить сравнительные результаты тестирования различных алгоритмов
    планирования. 

    Дальнейшее улучшение системы заключается в повышении разрешения симуляции,
    разработки удобного API для моделирования задач на разных языках
    программирования и с различными реализациями, например MapReduce-MPI
    \cite{mapreduce_mpi}.

    Предложенный метод симуляции показал себя с хорошей стороны, улучшая общую
    точность моделирования благодаря большему числу степеней свободы и, как
    следствие, более точному предсказанию времени работы и занимаемой памяти
    MapReduce приложением.

    \bigskip
    \bigskip
    \bigskip
    \bigskip
    \bigskip
    \bigskip

    Исходный код доступен на GitHub: \url{https://github.com/naorlov/thesis}
\end{document}


