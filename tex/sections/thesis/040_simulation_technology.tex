\documentclass[../diploma.tex]{subfile}


\begin{document}
    \section{Основные технологии для построения симуляторов}
    \label{sec:simulation_techonogies}

    \subsection{ROSS}

    ROSS (англ. акроним Rensselaer Optimistic Simulation System) \cite{ross}
    является основой для CODES, о котором речь пойдет ниже. Это
    дискретно-событийный симулятор, являющийся базовым фреймворком для
    построения различных. Он использует Time Wrap\footnote{\TODO{Как это перевести?}} -- технологию, предложенную в статьях авторов D. R. Jefferson и H. Sowizral \cite{}


    \subsection{CODES}

    CODES \cite{codes} -- фреймворк, используемый для построения симуляторов
    хранилища размера порядка единиц петабайтов. Внутри он использует ROSS для 
    

    \subsection{SimGrid}
    \label{sec:simulation_techonogies:subsec:simgrid}
    \textit{Simgrid}\cite{simgrid} -- фреймворк, упрощающий разработку
    симуляторов распределенных систем. Он предоставляет несколько разных
    интерфейсов для языков \textit{C++}, \textit{Python} и \textit{Java}. В
    данной работе использовался интерфейс \texttt{s4u} для C++, построенный на
    \textit{акторной} модели. 

    Основная идея акторной модели состоит в том, что симулируемая система
    представляется в виде набора \textit{акторов}, работающих на \textit{хостах}
    (англ. \textit{host} \TODO{Как это правильно перевести?}). Каждый актор
    представляется в виде функционального объекта. Акторы пишутся по образу и
    подобию моделируемой части системы, вызывая специальные методы API
    \footnote{англ. Application Programming Interface -- интерфейс
    программирования приложений}, реализующие общение актора с его окружением, в
    которое входит хост, на котором работает актор, а также все хосты в системе.
    
    Во время вызова методов API фреймворк сам создает \textit{событие} в
    терминох дискретно-событийной модели
    (\ref{sec:simulation_methods:subsec:discrete_event_modeling}). SimGrid берет
    на себя работу по вычислению количества реального времени, которое займет
    указанное событие. 
    
    Например, если актору необходимо прочитать определенное количество $s$
    информации с диска, то актор сначала получает информацию о своем хосте,
    затем он может выбрать диск, с которого будет происходить чтение данных.
    После этого актор вызывает метод API для чтения данных, передавая в качестве
    одного из параметров размер $s$. В этот момент фреймворк сам посчитает
    время, необходимое для чтения данных, исходя из скорости хоста, диска и
    запрошенного размера данных. 

    SimGrid зарекомендовал себя как качественный проект, на основе которого
    строится множество фреймворков моделирования, из которых часть описана в
    секции (\ref{sec:review}).

\end{document}


