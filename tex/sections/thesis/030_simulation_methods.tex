\documentclass[../diploma.tex]{subfile}


\begin{document}
    \section{Подходы к моделированию систем}
    \label{sec:simulation_methods}
    
    \textit{
        Примечание: здесь и далее термин <<симуляция>> означает по смыслу
        <<имитационное моделирование>>.
    }

    В современной литературе различают два подхода к моделированию компьютерных
    систем: \textit{дискретно-событийное} и \textit{аналитическое}
    моделирование.

    \subsection{Дискретно-событийное моделирование}
    \label{sec:simulation_methods:subsec:discrete_event_modeling}

    \textit{Дискретно-событийным моделированием} называется технология
    моделирования, при которой поведение системы во времени представляется в
    виде отдельных \textit{событий}. Событием в этой модели назвыается момент
    смены состояния системы во времени. Во время моделирования строится
    последовательность событий, представляющая из себя все изменения состояния
    системы. 
    
    Приведем пример событий, происходящих во время обработки запроса
    HTTP-сервером: 
    
    \begin{enumerate}
        \item Получение запроса от пользователя
        \item Подготовка обработчика запроса
        \item Обработка запроса
        \item Возврат ответа пользователю
    \end{enumerate}
    
    \subsubsection{Преимущества}

    Преимуществом такого моделирования можно назвать способность менять
    <<разрешение>> симуляции: в зависимости от требований, смуляция может
    происходить как на высоком, так и на низком уровне абстракции. При
    симуляции на высоком уровне повышается результирующая точность, но также
    повышается потребление памяти и время работы самого симулятора. При
    симуляции на низком уровне наоборот: понижается точность, потребление памяти
    и время работы симулятора.

    \subsubsection{Недостатки}

    В случае, если требуется получить высокую точностью симуляции, необходимо
    точно воссоздатьв все процессы моделируемой системы, что влечет за собой
    большие временные затраты на разработку такого симулятора.


    \subsection{Аналитическое моделирование}
    \label{sec:simulation_methods:subsec:analytical_modeling}
    \textit{Аналитическим моделированием} называется набор подходов к
    моделированию систем, предполагающий создание математического описания
    системы, позволяющего вычислить состояние этой системы. Ярким примером
    такого моделирования являются физические системы, описываемые набором
    дифференциальных уравнений. 


    \subsubsection{Преимущества}

    Главным преимуществом является простота такого симулятора -- достаточно
    найти зависимость моделируемого параметра (например, времени работы) от
    начальных условий системы, после чего получение целевого значения
    моделируемой системы является тривиальной задачей.

    \subsubsection{Недостатки}

    Главным недостатком является малая точность таких симуляторов (к примеру,
    симулятор YARNsim \cite{yarnsim} имеет меньшую точность, чем BTeHadoop2
    \cite{baseline_model}), а также неуниверсальность симулятора -- при
    изменении целевого значения необходимо выводить заново зависимость, тогда
    как при использовании дискретно-событийной модели добавление новой метрики
    является простым процессом -- система меняет состояние в строго определенных
    точках, до и после которых можно собирать глобавльное состояние системы.


\end{document}


