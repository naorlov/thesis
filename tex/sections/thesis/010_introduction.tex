\documentclass[../diploma.tex]{subfile}


\begin{document}
    \section{Введение}\label{sec:introduction}

    В современном мире MapReduce является основным фреймворком для обработки
    объемов данных гораздо больших, чем может обработать один компьютер.
    Представленная компанией Google статья \cite{google_mapreduce} 2004 года
    подробно описывает внутреннюю архитектуру такой системы. 

    Год к году количество производимых человечеством данных растет, что влечет
    за собой появление все большего количества кластеров по обработке данных. Но
    цена постройки таких кластеров высока -- стоимость центра обработки данных
    может достигать десятков миллионов долларов. Поэтому многие технологические
    компании хотят иметь возможность проводить симуляции своих вычислений с
    целью получения информации о необходимой архитектуре кластера для
    эффективного использования доступных средств.

    О популярности фреймворка говорит большое количество различных реализаций
    системы -- начиная от внутренних разработок IT-гигантов, как оставшихся
    закрытыми, как например в компании Яндекс \cite{yandex_yt}, так и
    предлагающих свои услуги пользователям, как компания Amazon
    \cite{amazon_emr}, заканчивая универсальной системой обработки больших
    данных -- Apache Hadoop \cite{apache_hadoop}, позволяющей использовать
    Apache Yarn \cite{apache_yarn} в качестве MapReduce клиента и Apache Hive
    \cite{apache_hive} для выполнения SQL-запросов над данными, и другие системы
    под эгидой Apache. 
    

    \subsection{Цели и задачи}\label{subsec:goals}

    В рамках дипломной работы были поставлены следующие задачи.

    \begin{enumerate}
        \item Реализация симулятора MapReduce кластера
        \item Моделирование выполнения MapReduce приложения на основе анализа
              данных времени выполнения и потребленной памяти пользовательской
              программой
    \end{enumerate}


    \subsection{Результаты}

    Итоговая реализация симулятора является достаточно точной для проведения
    экспериментов и работы (точность на уровне решений, представленных в
    литературном обзоре), но требует доработку в поиске модели времени
    выполнения (необходимо улучшение алгоритма обработки данных).

    \subsection{Структура работы}
    В главе \ref{sec:review} приведен обзор литературы, использованной при
    написании работы. Она разбита на четыре части: в части
    \ref{sec:review:subsec:whatismapreduce} дано описание парадигмы MapReduce. В части
    \ref{sec:review:subsec:simulation_methods} описаны методы симуляции систем.
    В части \ref{sec:review:subsec:simulation_technologies} описаны фреймворки,
    упрощающие реализацию симуляторов компьютерных систем. В части
    \ref{sec:review:subsec:existing} описаны существующие симуляторы MapReduce
    систем. 

    В главе \ref{sec:results} описана моя работа. В части
    \ref{sec:results:subsec:proposal} описано предлагаемое решение. В части
    \ref{sec:results:subsec:requirements} описаны требования, предъявленные к
    моему симулятору. В части \ref{sec:results:subsec:final} приведено
    техническое описание результата.

    В главе \ref{sec:testing} описаны результаты тестирования и сравнение моего
    результата с некоторыми симуляторами.

    В главе \ref{sec:final} дано заключение к работе.
    
\end{document}


