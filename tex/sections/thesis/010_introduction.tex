\documentclass[../diploma.tex]{subfile}


\begin{document}
    \section{Введение}\label{sec:introduction}

    В современном мире MapReduce является основным фреймворком для обработки
    объемов данных гораздо больших, чем может обработать один компьютер.
    Представленная компанией Google статья \cite{google_mapreduce} 2004 года
    подробно рассказывает о внутренней архитектуре такой системы. 

    Год к году количество производимых человечеством данных растет, что влечет
    за собой появление все большего количества кластеров по обработке данных. Но
    цена постройки таких кластеров высока -- стоимость ЦОД может достигать
    десятков миллионов долларов. Поэтому многие технологические компании хотят
    иметь возможность проводить симуляции своих вычислений с целью получения
    информации о необходимой архитектуре кластера для эффективного использования
    доступных средств.

    О популярности фреймворка говорит большое количество различных реализаций
    системы -- начиная от внутренних разработок IT-гигантов, как оставшихся
    закрытыми, как например в компании Яндекс \cite{yandex_yt}, так и ставших
    предлагающих свои услуги пользователям, как компания Amazon
    \cite{amazon_emr}, заканчивая универсальной системой обработки больших
    данных -- Apache Hadoop \cite{apache_hadoop}, позволяющей использовать
    Apache Yarn \cite{apache_yarn} в качестве MapReduce клиента и Apache Hive
    \cite{apache_hive} для выполнения SQL-запросов над данными, и другие системы
    под эгидой Apache. 
    

    \subsection{Цели и задачи}\label{subsec:goals}

    В рамках дипломной работы были поставлены следующие задачи.

    \begin{enumerate}
        \item Реализация симулятора MapReduce кластера
        \item Реализация алгоритмов анализа данных времени выполнения и
        потребленной памяти MapReduce программой
    \end{enumerate}

\end{document}


