\documentclass[../diploma.tex]{subfile}


\begin{document}
    \section{О MapReduce}\label{sec:whatismapreduce}
    
    \textit{MapReduce} это модель вычислений для обработки больших данных,
    представленная компанией Google в 2004 году \cite{google_mapreduce}. 
    
    Основная идея MapReduce состоит в разделении вычислений на два шага:
    \textit{Map} и \textit{Reduce}. На Map-шаге входные данные преобразовываются
    в пары ключ-значение $(key, value)$. На Reduce-шаге пары с одинаковым ключом
    группируются, и их значения агрегируются в финальное значение. 

    В статье авторы сильно опираются на свои наработки в области
    отказоустойчивых вычислений и отказоустойчивого хранения данных, в
    частности, MapReduce работает поверх GFS \cite{google_gfs} -- распределенной
    файловой системы. 

    Полностью работа MapReduce задачи, описанной в статье, выглядит следующим
    образом.    

    \begin{itemize}
        \item Пользователь отправляет задачу на исполнение в систему. Он может
        указать желаемые параметры исполнения, включающие в себя количество Map
        и Reduce заданий, размер входного и выходного блока и прочие параметры.

        \item Система запускает Map операцию.
        \begin{itemize}
            \item Делается запрос в GFS с целью узнать, на каких серверах
            находятся данные, являющиеся входными для Map-операции.
            \item На этих или близких к ним с точки зрения пропускной
            способности сети серверах запускаются Map-задачи. 
            \item По окончание Map-задачи запускается промежуточный и опицональный этап Combine. Цель этого этапа -- сделать предварительное объединение результатов.
        \end{itemize}

        \item Система выполняет Partition. 
        \begin{itemize}
            \item Строки, получившиеся на выходе Map задачи группируются по
            ключу и отправляются на сервера, на которых будет происходит
            выполнение Reduce-части вычислений. Параллельно идет сортировка этих данных для ускорения чтения на следующем шаге.
        \end{itemize}

        \item Система запускает Reduce-операцию.

        \begin{itemize}
            \item Строки с одинаковыми ключами обрабатываются вместе, их
            значения агрегируются в соответствии с пользовательским кодом.
            \item После обработки в GFS сохраняется вывод всех экземпляров
            Reduce-задачи.
        \end{itemize}
    \end{itemize}   

    Приведем примеры MapReduce задач. 
    
    \textit{Wordcount}. Классический пример MapRedcue задачи -- подсчитать
    количество слов в тексте. Здесь Map-операция читает входные данные и
    разбирает по словам, производя список пар (слово, 1). Затем Reduce-операция
    просто суммирует все значения, чтобы получить количетсво вхождений каждого
    слова.

    \textit{K-Means}. Задача машинного обучения, состоящая в разбиении множества
    точек на кластеры таким образом, что суммарное квадратичное отклонение точек
    кластеров от центров этих кластеров является минимальным. Здесь Map-операция
    считает для каждой точки ближайший центр кластера к этой точке, а
    Reduce-операция обновляет центры точек. Эти операции повторяются до сходимости.
    
    

\end{document}


