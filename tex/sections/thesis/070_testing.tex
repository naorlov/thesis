\documentclass[../diploma.tex]{subfile}


\begin{document}
    \section{Тестирование}
    \label{sec:testing}

    В качестве тестового приложения было выбрано приложение Wordcount
    (вычисление количества слов в исходном корпусе), исходные тексты Map и
    Reduce заданий указаны в листингах \ref{lst:wordcount_mapper} и
    \ref{lst:wordcount_reducer} соответственно. 

    Были симулированы следующие кластеры, состоящие из 1 и 2 хостов. Для
    каждого кластера количество входных данных изменялось от $10^2$ строк до
    $10^{6}$ строк в логарифмической шкале, средний размер строки примерно равен
    170 байт.

    Результаты тестирования при симуляции одного хоста приведены в таблице \ref{table:testing_single}

    \begin{figure}[H]
    \begin{center}
    \begin{tabular}{c|c|c|c|c|c}
        Размер входа              & 100   & 1000  & 10000 & 100000 & 1000000 \\ \hline
        Время работы MAP (реал)   & 0.010 & 0.013 & 0.020 & 0.208  & 3.079   \\
        Время работы MAP (пред)   & 0.009 & 0.013 & 0.020 & 0.172  & 2.824   \\
        Разница MAP               & 10\%  & 0 \%  & 0\%   & 17\%   & 12\%    \\ \hline
        Время работы RED (реал)   & 0.010 & 0.032 & 0.056 & 1.669  & 26.013  \\
        Время работы RED (пред)   & 0.049 & 0.047 & 0.052 & 1.997  & 29.898  \\
        Разница RED               & 390\% & 31\%  & 7\%   & 19\%   & 14\%    \\
    \end{tabular}
    \medskip
    \end{center}
    \caption{Результаты тестирования на 1 хосте}
    \label{table:testing_single}
    \end{figure}

    В таблице выше все времена представлены в секундах, размер входа -- в
    количестве строк. Разница считается по следующей формуле:

    \begin{equation}
        \text{\textit{difference}} = \frac{|t_{real} - t_{pred}|}{t_{real}}
    \end{equation}
    
    где $ t_{real} $ -- реальное время работы, $ t_{pred} $ -- предсказанное
    время работы.

    По итогам моделирования одного хоста можно говорить об удовлетворительной
    предсказательной способности метода моделирования исполнения одной задачи.


    Результаты тестирования на двух хостах приведены в таблице
    \ref{table:testing_double}.

    \begin{figure}[H]
    \centering
    \begin{tabular}{c|c|c|c|c|c}
        Размер входа          & 100    & 1000   & 10000  & 100000  & 1000000 \\ \hline
        Время работы (пред)   & 23.006 & 23.013 & 24.020 & 26.208  & 48.079   \\
        Время работы (реал)   & 21.009 & 22.013 & 26.020 & 28.172  & 46.824   \\
        Разница               & 9\%    & 4\%    & 8\%    & 7\%     & 4\%
    \end{tabular}
    \caption{Результаты тестирования на 2 хостах}
    \label{table:testing_double}
    \end{figure}

    По итогам тестирования можно увидеть лучшую точность совокупного
    моделирования всей системы. 

\end{document}


