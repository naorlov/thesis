\documentclass[conference]{IEEEtran} \IEEEoverridecommandlockouts
% The preceding line is only needed to identify funding in the first footnote.
% If that is unneeded, please comment it out.
\usepackage{amsmath,amssymb,amsfonts} \usepackage{algorithmic}
\usepackage{graphicx} \usepackage{textcomp} \usepackage{xcolor}
\usepackage{icomma}
% \def\BibTeX{{\rm B\kern-.05em{\sc i\kern-.025em b}\kern-.08em
% T\kern-.1667em\lower.7ex\hbox{E}\kern-.125emX}}


\usepackage[backend=bibtex, sorting=none]{biblatex}

\addbibresource{bibliography.bib}
\bibliographystyle{ieeetr}

\newcommand*{\TODO}[1]{{\color{red}{TODO: \textit{#1}}}}


\newtheorem{oldlem}{Лемма}[section] \newenvironment{lemma}[1][]
{\begin{oldlem}[#1]} {\end{oldlem}}

\newtheorem{olddefi}[oldlem]{Definition} \newenvironment{definition}[1][]
{\begin{olddefi}\normalfont} {\end{olddefi}}


\begin{document} \title{DRAFT: Simulating execution of MapReduce applications}

    \author{ 
        \IEEEauthorblockN{Nikita Orlov} 
        \IEEEauthorblockA{ 
            \textit{HSE University} \\ 
            Moscow, Russia \\ 
            naorlov\_1@edu.hse.ru 
        } 
    }

    \maketitle

    \begin{abstract} 
    A new method of MapReduce simulation is proposed combining
    event-discrete simulatros and analytical models, that are built on the fly
    from data, gathered with a help of performance benchmarks.
    \end{abstract}

    \begin{IEEEkeywords} 
        MapReduce, simulation, analytical model, event discrete simulator
    \end{IEEEkeywords}

    \section{Introduction}
    In 2004 Google released a paper \cite{google_mapreduce} on the new data
    processing framework called MapReduce. It is capable of processing a
    significant amount of data that is not processed on a single machine with
    the help of two processing stages: map and reduce. On the map stage, input
    data is processed, and tuples consisting of keys and values are outputted
    to reduce stage, where input data is grouped by key and processed as a
    single batch of data. Authors claim that a vast majority of algorithms
    could be written using MapReduce jobs -- a Map step followed by Reduce
    step, possibly a number of these jobs. Many interfaces for MapReduce have
    been developed over a past decade -- Apache Hadoop \cite{apache_hadoop}
    (distributed application execution engine and MapReduce engine), Apache
    Hive \cite{apache_hive} (a SQL-like data manipulation language), Amazon
    Elastic MapReduce \cite{amazon_emr} (cloud MapReduce solution for on-demand
    data processing) are popular ones. MapReduce is also used as a programming
    paradigm on top of other storage solutions, like a NoSQL database Apache
    CouchDB \cite{apache_couchdb}, that can execute MapReduce tasks over stored
    data.

    Since Google released the paper \cite{google_mapreduce} on MapReduce, a
    considerable amount of research was put on developing systems that store,
    process, and transform a significant amount of data. Today, big companies
    are producing hundreds of terabytes of data on a daily basis. With the help
    of MapReduce and related work, this data can be processed in a matter of
    minutes, giving businesses and individuals an opportunity to perform more
    complex data analysis. 

    Several papers are released on this topic, but they do not provide any
    code, or the code is not updated for a couple of years and is outdated.
    Having such a tool that can accurately predict execution time can be useful
    in different use cases. For example, one could develop several scheduling
    algorithms and quickly test them in simulation. Another possible use is to
    determine the best cluster configuration for new MapReduce cluster -- one
    could take existing tasks and look through various cluster configurations.

    This study is aimed to build a framework for accurate simulation and
    prediction of execution times for a user-provided MapReduce applicaiton on
    a homogenous MapReduce cluster. We make no assumptions about Map and Reduce
    stage implementations, their input and output data size and distribution.
    However, cluster is assumed to be homogenous (i.e. every server has the
    same number of CPU cores, RAM size, I/O configuration), since these
    clusters are both cheaper to build and operate when compared to
    heterogeneous environments and more popular among existing MapReduce
    installations.

    \section{Literature review} 
    \label{literature_review} 

    For the past decade, several studies have proposed techniques to model the
    execution of MapReduce applications. They can be differentiated by used
    modelling method. 

    \textit{Discrete-event simulation} is a method that represents a system's
    operation as an ordered list of events. Every event contains its timestamp,
    representing \textit{event time} at which event took place. The simulation
    engine algorithm is straightforward: it sets the current time to 0 and
    iterates over a set of events, on each step advancing current time. When
    the simulation is completed, the engine reports a total simulation time, as
    well as several runtime statistics. While using this method, researches can
    balance between time took by simulator and prediction error. Describing
    more aspects of a system leads to an increase in the overall simulation run
    time, while increasing simulation accuracy, and in the opposite, simulation
    time can be reduced by decreasing the temporal resolution of simulation
    (mainly by reducing the number of events, as well as the number of
    entities, internal framework processes).

    ROSS \cite{ross} (Rensselaer Optimistic Simulation System) is an example of
    parallel discrete-event simulator. It is used by CODES \cite{codes}, an I/O
    and storage emulator, that allows building an accurate model of exascale
    storage solutions. These frameworks are used in several simulators (i.e.,
    BTeHadoop2 \cite{baseline_model}).

    SimGrid \cite{simgrid} is a low-level simulation framework for simulating
    complex multi-server environment. It uses its own implementation of a
    parallel discrete-event simulator as described in book by Fujimoto, Richard
    \cite{fujimoto_parallel_1990}.

    \textit{Analytical modelling} is a method that uses a set of performance
    equations that can predict execution times for different parts of the
    MapReduce system, including mappers and reducers. In most of the cases,
    these equations are built from user defined constants on predefined
    equations. Some of the simulators, like BTeHadoop2 \cite{baseline_model},
    build these equations based on performance benchmarks of supplied MapReduce
    job description.

    Some existing simulators are based on \textit{discrete-event modelling},
    like Mumak \cite{mumak}, SimMR \cite{simmr}, MRperf \cite{mrperf}, MRSG
    \site{mrsg}, YarnSim \cite{yarnsim} and MRSim \cite{mrsim}. They all share
    the same principle: the core of the MapReduce system is simulated, where
    user-supplied mappers and reducers characteristics are taken either from
    execution benchmarks or from the user-provided information. MRPerf assumes
    that every job is a single mapper-reducer pair, and their corresponding
    outputs size and execution time are dependent only on input size and not
    the contents of input data. 

    In contrary, Song et al. \cite{song} proposed a method that is based on
    \textit{an analytical modelling} technique. They gather some metadata about
    the supplied MapReduce job and pass them to predefined model that estimates
    execution time. MRSim uses this technique while modelling mapper and
    reducer job, relying on user-provided constants output sizes and execution
    cost. YARNSim uses a predefined model for the map and reduce stages,
    providing a configurable parameter for corresponding input sizes. SLS
    \cite{sls} is an Apahce YARN\cite{apache_yarn} scheduler simulator capable
    of experimenting with various scheduling Another way to differentiate is to
    consider a system design and simulation scope: Mumak and SLS\cite{sls} are
    scheduler simulators, mainly used to study different scheduling algorithms,
    and MRperf, MRSG and MRsim are task execution engine simulators, providing
    a tool for studying different cluster architectures.


    \section{Methods} 

    This project uses a mix of two approaches given in Section 
    \ref{literature_review}.

    First, the application's mapper and reducer programs are tested to obtain
    performance statistics. The program is run ten times. Each time several
    parameters are recorded, including but not limited to total execution time,
    maximum memory use, input, and output data size.

    Then this data is passed to the Morpheus analytics system. It builds
    complexity functions for a given executable file. Complexity function for a
    given executable program is a function that maps input size in bytes and
    rows to overall execution time, RAM usage, output file size in bytes and
    rows. Then, all the gathered data is split between a training set and a
    validation set. For the given assumptions in the previous section, the
    Lambda algorithm is used.

    Lambda algorithm is proposed to find an analytical model for input/output
    parameters relationship. We assume that all tested programs are mappers and
    reducers, and their complexity functions can be written as follows:
    
    \begin{equation} 
        f(n) = \sum_{i = 0}^d a_in^i + \sum_{i=0}^d b_in^i\log(n)
    \end{equation} 
    
    where $a_i, b_i \in \mathbb{R}$, $d_l, d_n$ - degree of freedom for linear 
    and logarithmic part. For any reasonable purposes we can assume 
    $d_l, d_n \leqslant 10$. 

    These assumptions come from mapper's and reducer's primary purpose. Mapper
    reads data and outputs corresponding key-value tuples, therefore it has to
    at least read all input data and produce output, contributing to $O(n)$
    complexity.  During the data scan, the mapper can build a dictionary to
    check for already lookup data, or sort data internally, contributing to
    $O(n)\cdot O(\log n)$ complexity.  he constant factor is added to support
    mappers with external dictionary lookups and other similar tasks, that are
    not dependent on mapper's input data size. Reducer reads mapper's output
    grouped by key and outputs final value for any given key.  It also can
    perform similar operations with constant, logarithmic, linear and $O(n^k
    \cdot \log n)$ time-space complexity (i.e. input data or external 
    dictionary lookups, sorting, multiplying output rows).

    Other methods could be investigated for building an analytical model from
    gathered data, for example, Lagrange polynomial, but this is not a suitable
    solution for building a function that could accurately predict execution
    times for input sizes beyond the maximum observed point. For example, if we
    take a quadratic function, sample values of the function at six different
    points and calculate Lagrange polynomial for these points, we will get a
    6-degree polynomial incapable of predicting original function at other
    points. 

    Within the proposed method, all possible combinations of the base
    polynomial are used to fit a linear regression on their associative
    coefficients. Then a model with the best $R^2$ metrics on validation data
    is chosen. This model is then treated as a complexity function.

    Then the analytical model is used in simulation agents of
    SimGrid\cite{simgrid} framework to tell precisely the amount of operations
    needed to be performed, and the amount of data is produced.

    Simulation agents used in SimGrid are programmed to perform all execution
    steps for a given MapReduce job. Apache Hadoop framework is chosen as an
    architecture simulated due to the fact that this framework is open-source
    and well documented, so we can accurately model its components. 


    \section{Results anticipated}

    It is hoped that the end result of this study will be a complete MapReduce
    simulation framework, that will be capable of evaluating the performance of
    a MapReduce applicaiton executed on large data processing clusters. The
    resulting toolchain will be compared and tested against other available
    MapReduce simulators to reveal any possible advantages and disadvantages of
    a proposed system. Finally, a number of popular MapReduce applications will
    be used to compare the overall system's accuracy and real execution time on
    a real cluster. 

    \section{Conclusion}

    
    \printbibliography

    \vspace{10 mm}

    \begin{flushright}
        Word count: 
    \end{flushright}

\end{document}


