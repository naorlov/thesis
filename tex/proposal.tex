\documentclass[conference]{IEEEtran}
\IEEEoverridecommandlockouts
% The preceding line is only needed to identify funding in the first footnote. If that is unneeded, please comment it out.
\usepackage{amsmath,amssymb,amsfonts}
\usepackage{algorithmic}
\usepackage{graphicx}
\usepackage{textcomp}
\usepackage{xcolor}
\usepackage{icomma}
% \def\BibTeX{{\rm B\kern-.05em{\sc i\kern-.025em b}\kern-.08em
%     T\kern-.1667em\lower.7ex\hbox{E}\kern-.125emX}}


\usepackage[backend=bibtex]{biblatex}

\addbibresource{bibliography}

\newcommand*{\TODO}[1]{{\color{red}{TODO: \textit{#1}}}}


\newtheorem{oldlem}{Лемма}[section]
\newenvironment{lemma}[1][]
    {\begin{oldlem}[#1]}
    {\end{oldlem}}

\newtheorem{olddefi}[oldlem]{Definition}
\newenvironment{definition}[1][]
    {\begin{olddefi}\normalfont}
    {\end{olddefi}}


\begin{document}
    \title{DRAFT: Framework for Simulation of Distributed Systems}

    \author{
        \IEEEauthorblockN{Nikita Orlov}
        \IEEEauthorblockA{
            \textit{Higher School of Economics} \\
            Moscow, Russia \\
            naorlov\_1@edu.hse.ru
        }
    }

    \maketitle

    \begin{abstract}
        \TODO{Abstract}
    \end{abstract}

    \begin{IEEEkeywords}
        \TODO{Keywords}
    \end{IEEEkeywords}

    \section{Introduction}
    \subsection{Definitions of key terms}
        \begin{definition}[kdf]
            kek
        \end{definition}
    \subsection{Background}

    Since Google released a paper on MapReduce framework \cite{google_mapreduce}, a considerable amount of research was put on storing, processing and transformation of a lot amount of data. Today, big companies are producing hundreds of terabytes of data on daily basis.  

    \subsection{Problem Statement}

    Today, building bigdata processing cluster is not cheap. In order, to achieve maximum performance of MapReduce framework, one should have some hundreds of servers, build a networking solution for them, install all this hardware into the datacenter or build one and finally put in some people to maintain the system. 
    
    The specific objective to this study is to build an open-source simulation platform to perform a cluster simulation with given hardware and software parameters. 

    Several papers are released on this topic, but they do not provide any code \cite{baseline_model} or did not updated for a couple of years \cite{yarnsim} and are outdated.

    \subsection{Professional Significance}

    \subsection{Delimitation of the study}

    This study is aimed to simulate cluster with configurations that meet followinng requirements:
    \begin{enumerate}
        \item All servers in cluster are the same configuration.
        \item Servers are grouped into multilayer network
        \item 
    \end{enumerate}

    \section{Literature review}
        Up to now, a number of studies have proposed some techniques to build a MapReduce simulation models. 


    \section{Methods}
        \TODO{Methods}

    \section{Results anticipated}
        \TODO{Results anticipaed}

    \section{Conclusion}
        \TODO{Conclusion}

    \printbibliography

\end{document}