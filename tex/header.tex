%% Начало шапки
%% Настройка поддержки русского языка
\usepackage{cmap}                   % Поиск по кириллице
\usepackage{mathtext}               % Кириллица в формулах
\usepackage[T1,T2A]{fontenc}        % Кодировки шрифтов
\usepackage[utf8]{inputenc}         % Кодировка текста
\usepackage[english,russian]{babel} % Подключение поддержки языков

%% Отступы между абзацами и в начале абзаца
\setlength{\parindent}{15pt}
\setlength{\parskip}{\medskipamount}

%% Настройка размеров полей
\usepackage[top=0.7in, bottom=0.75in, left=1.2in, right=1.2in]{geometry}

%% Графика
\usepackage[pdftex]{graphicx}
\graphicspath{{images/}}

%% Математические пакеты
\usepackage{mathtools}              % Тот же amsmath, только с некоторыми поправками
\usepackage{amssymb}                % Математические символы
\usepackage{amsthm}                 % Оформление теорем
\usepackage{amstext}                % Текстовые вставки в формулы
\usepackage{amsfonts}               % Математические шрифты
\usepackage{icomma}                 % "Умная" запятая: $0,2$ --- число, $0, 2$ --- перечисление
\usepackage{enumitem}               % Для выравнивания itemize (\begin{itemize}[align=left])
\usepackage{array}                  % Таблицы и матрицы
\usepackage{multirow}

%% Алгоритмические пакеты и их настройки
\usepackage{algcompatible}
\usepackage{algorithm}             % Шапка алгоритма
\usepackage{algorithmicx}           % Написание алгоритмов
\usepackage[noend]{algpseudocode}   % Написание псевдокода; убраны end
\usepackage{listings}               % Для кода на каком-либо языке программиования
\renewcommand{\algorithmicrequire}{\textbf{Ввод: }}              % Ввод
\renewcommand{\algorithmicensure}{\textbf{Вывод: }}              % Вывод
\floatname{algorithm}{Алгоритм}                                 % Название алгоритма
\renewcommand{\algorithmiccomment}[1]{\hspace*{\fill}\{// #1\}} % Комментарии
\newcommand{\algname}[1]{\textsc{#1}}                           % Вызов функции в алгоритме

\newcommand*{\hm}[1]{#1\nobreak\discretionary{}
	{\hbox{$\mathsurround=0pt #1$}}{}}

%% Шрифты
\usepackage{euscript}               % Шрифт Евклид
\usepackage{mathrsfs}               % \mathscr{}

%% Графика
\usepackage[pdftex]{graphicx}       % Вставка картинок
\graphicspath{{images/}}            % Стандартный путь к картинкам
% \usepackage{tikz}                   % Рисование всего
% \usetikzlibrary{arrows.meta}
% \usepackage{pgfplots}               % Графики
% \usetikzlibrary{calc, matrix, intersections, fit, patterns}

%% Прочие пакеты
\usepackage{indentfirst}                    % Красная строка в начале текста
\usepackage{epigraph}                       % Эпиграфы
\usepackage{fancybox,fancyhdr}              % Колонтитулы
\usepackage[
colorlinks = false, hidelinks
]{hyperref}   % Ссылки
\usepackage{titlesec}                       % Изменение формата заголовков
\usepackage{tocloft}						% Изменение формата содержания
\usepackage[normalem]{ulem}                 % Для зачёркиваний
\usepackage[makeroom]{cancel}               % И снова зачёркивание (на этот раз косое)
\usepackage{subfiles}						% \subfile

% Графики
\usepackage{pgfplots}
% \pgfplotsset{compat=1.12}


%% Прочее
\mathtoolsset{showonlyrefs=true}        % Показывать номера только у тех формул,
% на которые есть \eqref{} в тексте.
\renewcommand{\headrulewidth}{1.8pt}    % Изменяем размер верхнего отступа колонтитула
\renewcommand{\footrulewidth}{0.0pt}    % Изменяем размер нижнего отступа колонтитула

%Прочее
\usepackage{forest} % Деревья

%% Для колонтитула
\def\head{
{\it \small  НИУ ВШЭ $\cdot$ Факультет компьютерных наук $\cdot$
Прикладная математика и информатика}
}


\renewcommand{\Re}{\mathrm{Re\:}}
\renewcommand{\Im}{\mathrm{Im\:}}
\newcommand{\Arg}{\mathrm{Arg\:}}
\renewcommand{\arg}{\mathrm{arg\:}}
\newcommand{\Mat}{\mathrm{Mat}}
\newcommand{\M}{\mathrm{M}}
\newcommand{\id}{\mathrm{id}}
\newcommand{\isom}{\xrightarrow{\sim}}
\newcommand{\leftisom}{\xleftarrow{\sim}}
\newcommand{\Hom}{\mathrm{Hom}}
\newcommand{\Ker}{\mathrm{Ker}\:}
\newcommand{\rk}{\mathrm{rk}\:}
\newcommand{\diag}{\mathrm{diag}}
\newcommand{\ort}{\mathrm{ort}}
\newcommand{\pr}{\mathrm{pr}}
\newcommand{\vol}{\mathrm{vol\:}}
\newcommand{\Tr}{\mathrm{tr\:}}
\newcommand{\sgn}{\mathrm{sgn\:}}
\newcommand{\al}{\alpha}
\usepackage{calculator}

\newcommand{\then}{\Rightarrow}

\newcounter{QuestionCounter}
\stepcounter{QuestionCounter}

\newenvironment{question}{
	\textbf{Вопрос \arabic{QuestionCounter}.}
	\itshape
}{
}
\newenvironment{answer}{
	\par
	\noindent
	\stepcounter{QuestionCounter}
	%	\ignorespaces
}{
	\sep
}
\newcommand{\sep}{
	\begin{center}
		\line(1,0){450}
	\end{center}	
}


\newcounter{ProblemCounter}

\newenvironment{task}{
	\stepcounter{ProblemCounter}
	\noindent 
	\textbf{Задача \arabic{ProblemCounter}.}
	\itshape
%	\ignorespaces
}{
}
\newenvironment{solution}{
	\par
	\noindent
%	\ignorespaces
	\textbf{Решение.}
}{
        \qed
        \sep
}


\renewcommand{\mod}{\text{mod}}
\renewcommand{\iff}{\Leftrightarrow}
\newcommand{\from}{\colon}
\renewcommand{\emptyset}{\varnothing}
%% Определения
\newtheorem{oldlem}{Лемма}[ProblemCounter]
\newenvironment{lemma}[1][]
    {\begin{oldlem}[#1]}
    {\end{oldlem}}

\newtheorem{oldprep}[oldlem]{Предложение}
\newenvironment{preposition}[1][]
    {\begin{oldprep}[#1]\normalfont}
    {\end{oldprep}}

\newtheorem{oldtheor}[oldlem]{Теорема}
\newenvironment{theorem}[1][]
    {\begin{oldtheor}[#1]}
    {\end{oldtheor}}

\newtheorem{oldcorol}[oldlem]{Следствие}
\newenvironment{corollary}[1][]
    {\begin{oldcorol}[#1]\normalfont}
    {\end{oldcorol}}

\newtheorem{oldrem}[oldlem]{Замечание}
\newenvironment{remark}[1][]
    {\begin{oldrem}[#1]\normalfont}
    {\end{oldrem}}

\newtheorem{olddefi}[oldlem]{Определение}
\newenvironment{definition}[1][]
    {\begin{olddefi}[#1]\normalfont}
    {\end{olddefi}}

\newtheorem{oldstat}[oldlem]{Утверждение}
\newenvironment{statement}[1][]
    {\begin{oldstat}[#1]}
    {\end{oldstat}}

\newtheorem{oldex}[oldlem]{Пример}
\newenvironment{example}[1][]
    {\begin{oldex}[#1]\normalfont}
    {\end{oldex}}

\newtheorem{oldprop}[oldlem]{Свойство}
\newenvironment{property}[1][]
    {\begin{oldprop}[#1]}
    {\end{oldprop}}

\newcommand{\Z}{\mathbb{Z}}
\newcommand{\N}{\mathbb{N}}
\newcommand{\Q}{\mathbb{Q}}
\newcommand{\R}{\mathbb{R}}
\newcommand{\D}{\mathbb{D}}
\renewcommand{\C}{\mathbb{C}}
\renewcommand{\P}{\mathbb{P}}
\renewcommand{\L}{\mathscr{L}}
\renewcommand{\epsilon}{\varepsilon}
\renewcommand{\phi}{\varphi}
\renewcommand{\qedsymbol}{\textbf{[:|||:]}}
\newcommand{\e}{\mathbb{e}}
\renewcommand{\l}{\lambda}
\newcommand{\E}{\mathbb{E}}
\def\eps{\varepsilon}
\def\limref#1#2{{#1}\negmedspace\mid_{#2}}
\newcommand{\vvector}[1]{\begin{pmatrix}{#1}_1 \\\vdots\\{#1}_n\end{pmatrix}}
\renewcommand{\vector}[1]{({#1}_1, \ldots, {#1}_n)}

\definecolor{Gray}{gray}{0.9}
\newcolumntype{g}{>{\columncolor{Gray}}c}

\newenvironment{proof_cor}
    {\textit{Доказательство корректности.}}
    {\qed \par}

\renewenvironment{time}
    {\textit{Время работы.}}
    {\qed \par}

\renewcommand{\cftsecleader}{\cftdotfill{\cftdotsep}} 			% точки в содержании
\renewcommand{\cftsecpresnum}{\begin{lrbox}{0}}					% убираем цифру
	\renewcommand{\cftsecaftersnum}{\end{lrbox}}
\setlength{\cftsecnumwidth}{22pt}								% отступ от края страницы


\def\leftbracket{[}
\def\rightbracket{]}

\tikzset{my arrow/.style={
		blue!60!black,
		-latex
	}
}

\newcommand*{\TODO}[1]{{\color{red}{TODO: \textit{#1}}}}



\renewcommand{\leq}{\leqslant}
\renewcommand{\geq}{\geqslant}

\usepackage{titlesec}
\titleformat{\chapter}
{\LARGE\bfseries}
{}
{0pt}
{}
\titleformat{\section}
{\centering\vspace{1cm}\LARGE\bfseries} % Стиль заголовка
{}                        % префикс
{0pt}                     % Расстояние между префиксом и заголовком
{}                        % Как отображается префикс
\titleformat{\subsection} % Аналогично для \subsection
{\Large\bfseries}
{}
{0pt}
{}
\titleformat{\subsubsection} % Аналогично для \subsubsection
{\large\bfseries}
{}
{0pt}
{}

\DeclareMathOperator{\dom}{dom}
\newcommand{\grad}{\bigtriangledown}


\usepackage{dsfont}

\setlength\parindent{0pt}